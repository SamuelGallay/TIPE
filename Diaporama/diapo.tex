\documentclass[aspectratio=43]{beamer}

\usepackage[T1]{fontenc}

\usepackage{xcolor}
\usepackage{diagbox}

\usepackage[sfdefault, light]{FiraSans}
\usepackage{fancyvrb}
%\usepackage{minted}
\usepackage{tikz}
\usetikzlibrary{shapes}
\usepackage[french]{babel}

\usecolortheme{crane}

\title{Programmation logique et contrôle d'accès}

\author{\textsc{Samuel Gallay}}

\date{\today}

\begin{document}

\frame{\titlepage}


\begin{frame}
\frametitle{Contrôle d'accès : motivation}

\begin{center}
\begin{tikzpicture}
  \node[draw, ellipse] (S) at (0,0) {Sujet};
  \node[draw, ellipse] (O) at (8,2) {Objet};
  \node[draw, diamond, aspect=2.5] (C) at (6,0) {Contrôleur};
  \node[draw, rectangle,rounded corners=3] (R) at (2.5, 0) {Requête};
  \node[draw, rectangle,rounded corners=3] (D) at (6, -2) {Décision};
  \node[draw, rectangle] (A) at (4, 2) {Autorisations};


  \draw[->,>=latex](S) -- (R);
  \draw[->,>=latex](R) -- (C);
  \draw[->,>=latex](O) -- (C);
  \draw[->,>=latex](C) -- (D);
  \draw[->,>=latex](A) -- (C);
\end{tikzpicture}
\end{center}
\end{frame}


\begin{frame}
\frametitle{Contrôle d'accès : motivation}

\begin{center}
\begin{tikzpicture}
  \node[draw, ellipse] (S) at (0,0) {Sujet};
  \node[draw, ellipse] (O) at (8,2) {Objet};
  \node[draw, diamond, aspect=2.5] (C) at (6,0) {Contrôleur};
  \node[draw, rectangle,rounded corners=3] (R) at (2.5, 0) {Requête};
  \node[draw, rectangle,rounded corners=3] (D) at (6, -2) {Décision};
  \node[draw, rectangle] (A) at (4, 2) {Autorisations};

  \node (u) at (0.2,1) {\textcolor{purple}{utilisateur}};
  \node (ac) at (2, -1) {\textcolor{purple}{accès à un fichier}};
  \node (f) at (8, 1) {\textcolor{purple}{fichier}};

  \draw[->,>=latex](S) -- (R);
  \draw[->,>=latex](R) -- (C);
  \draw[->,>=latex](O) -- (C);
  \draw[->,>=latex](C) -- (D);
  \draw[->,>=latex](A) -- (C);

  \draw[-,>=latex, purple](u) -- (S);
  \draw[-,>=latex, purple](ac) -- (R);
  \draw[-,>=latex, purple](f) -- (O);
\end{tikzpicture}
\end{center}
\end{frame}


\begin{frame}
\frametitle{Matrice de contrôle d'accès : limitations}

\begin{center}
\noindent
\begin{tabular}{|c|llll|}\hline
\backslashbox{Fichier}{Utilisateur}&Alice&Bob&Charlie&...\\\hline
Fichier 1&1&1&0&...\\
Fichier 2&0&1&1&...\\
Fichier 3&1&0&0&...\\
...&...&...&...&\\\hline
\end{tabular}
\end{center}

\bigskip

\begin{alertblock}{Inconvénients}
  \begin{itemize}
    \item taille de la matrice
    \item suppression des autorisations
    \end{itemize}
\end{alertblock}

\end{frame}



\begin{frame}
  \frametitle{Arborescence de fichiers et groupes d'utilisateurs}
\begin{columns}

\column{0.5\textwidth}
\begin{center}
  \begin{tikzpicture}[sibling distance=3em, level distance=3em, every node/.style = {shape=rectangle, rounded corners, align=center}]
  \node at (0, 0) {$g_{A}$}
    child { node {$u_{1}$} }
    child { node {$g_{B}$}
      child { node {$g_{C}$}
        child { node {$u_{3}$} }
        child { node {$u_{4}$} }
        child { node {$u_{5}$} } }
      child { node {$u_{2}$} }};
    \node at (2, 0) {$g_{D}$}
    child { node {$u_{6}$} };
\end{tikzpicture}
\end{center}

$u_{i}$ : utilisateur

$g_{i}$ : groupe

\vspace{1em}

$appartient_{gr}(u_{3}, g_{C})$

$inclus_{gr}(g_{B}, g_{A})$

\column{0.5\textwidth}
\begin{center}
  \begin{tikzpicture}[sibling distance=3em, level distance=3em,
    every node/.style = {shape=rectangle, rounded corners, align=center},
    level 2/.style={sibling distance=6em},
    level 3/.style={sibling distance=3em}]
    \node at (0, 0) {$d_{A}$}
    child { node {$f_{1}$} }
    child { node {$d_{B}$}
      child { node {$d_{C}$}
        child { node {$f_{2}$} }
        child { node {$f_{3}$} }
       }
       child { node {$d_{D}$}
         child { node {$f_{4}$} }
         child { node {$f_{5}$} }  }};
  \end{tikzpicture}
\end{center}

$f_{i}$ : fichier

$d_{i}$ : dossier

\vspace{1em}

$appartient_{dos}(f_{4}, d_{D})$

$inclus_{dos}(d_{C}, d_{B})$

  \end{columns}
\end{frame}


\begin{frame}
  \frametitle{Arborescence de fichiers et groupes d'utilisateurs}
\begin{columns}

\column{0.5\textwidth}
\begin{center}
  \begin{tikzpicture}[sibling distance=3em, level distance=3em, every node/.style = {shape=rectangle, rounded corners, align=center}]
  \node at (0, 0) {$s_{1}$}
    child { node {$s_{2}$} }
    child { node {$s_{3}$}
      child { node {$s_{4}$}
        child { node {$s_{5}$} }
        child { node {$s_{6}$} }
        child { node {$s_{7}$} } }
      child { node {$s_{8}$} }};
    \node at (2, 0) {$s_{9}$}
    child { node {$s_{10}$} };
\end{tikzpicture}
\end{center}

$s_{i}$ : sujet

\vspace{1em}

$sous\_sujet(s_{2}, s_{1})$

\column{0.5\textwidth}
\begin{center}
  \begin{tikzpicture}[sibling distance=3em, level distance=3em,
    every node/.style = {shape=rectangle, rounded corners, align=center},
    level 2/.style={sibling distance=6em},
    level 3/.style={sibling distance=3em}]
    \node at (0, 0) {$o_{1}$}
    child { node {$o_{2}$} }
    child { node {$o_{3}$}
      child { node {$o_{4}$}
        child { node {$o_{5}$} }
        child { node {$o_{6}$} }
       }
       child { node {$o_{7}$}
         child { node {$o_{8}$} }
         child { node {$o_{9}$} }  }};
  \end{tikzpicture}
\end{center}

$o_{i}$ : objet

\vspace{1em}

$sous\_objet(o_{2}, o_{1})$

  \end{columns}
\end{frame}




\begin{frame}
\frametitle{Les règles d'autorisation}
\begin{itemize}
  \item Règles simples :
      $$autorise(s_{i}, o_{j})\textrm{, ou }sous\_sujet(s_{i}, s_{j})$$
  \item Propagation des autorisations :
    \begin{align*}
      &\forall U, \forall G, \forall D, \\
      &sous\_sujet(U, G) \wedge autorise(G, D) \Rightarrow autorise(U, D)
      \end{align*}
    \begin{align*}
      &\forall F, \forall D, \forall U, \\
      &sous\_objet(F, D) \wedge autorise(U, D) \Rightarrow autorise(U, F)
      \end{align*}
\end{itemize}

\end{frame}

\begin{frame}
\frametitle{Objectifs}

Écrire un système permettant :
\begin{enumerate}
  \item de représenter ce type de règles
  \item de répondre à des questions comme :
    \begin{itemize}
      \item \it{Alice peut-elle accéder au fichier A ?}
      \item \it{Qui est-ce qui peut accéder au fichier A ?}
    \end{itemize}
\end{enumerate}
\end{frame}

\begin{frame}
\frametitle{Logique du premier ordre, ou calcul des prédicats}

\begin{itemize}
\item Des prédicats, des variables, des fonctions, les connecteurs logiques $\vee$ et $\wedge$, et les quantificateurs $\exists$ et $\forall$.

\item Les règles sont des \textit{formules bien définies} dont toutes les variables sont liées (formules clauses).

\item Mise sous forme prénexe.

\item Skolémisation.

\item Mise sous forme normale conjonctive.
\end{itemize}
\end{frame}

\begin{frame}
\frametitle{Logique du premier ordre, ou calcul des prédicats}
  Les règles sont maintenant toutes de la forme :
  $$\forall X_{1}...X_{s},\bigwedge_{i=1}^{n}\left(\bigvee_{j=1}^{k}L_{i,j}\right)$$
  En ne notant plus les quantificateurs, et en séparant en $n$ clauses, on se réduit à la forme :
  $$\bigvee_{i=1}^{k}L_{i}$$ où les $L_{i}$ sont des littéraux positifs ou négatifs.
\end{frame}

\begin{frame}
  \frametitle{Restriction aux clauses définies}
  \begin{alertblock}{Attention}
    Perte de généralité par rapport à la logique du premier ordre : existence d'algorithmes efficaces sur les clauses de Horn.
  \end{alertblock}

  Exactement un littéral positif :
  $$\left(\bigvee_{i=1}^{n}\neg P_{i}\right)\vee{Q} \equiv \bigwedge_{i=1}^{n}P_{i}\Rightarrow Q$$
  Si $n=0$ : un fait, Q toujours vrai.

  \vspace{1em}

  Notation Prolog :\\
  \texttt{q(f1(X), ...) :- p1(f2(Y), ...), ..., pn(fk(Z), ...).}
\end{frame}


\begin{frame}[fragile]
  \frametitle{Exemple de programme Prolog}
\begin{Verbatim}
apprend(eve, mathematiques).
apprend(benjamin, informatique).
apprend(benjamin, physique).
enseigne(alice, physique).
enseigne(pierre, mathematiques).
enseigne(pierre, informatique).

étudiant_de(E,P):-apprend(E,M), enseigne(P,M).

étudiant(E, pierre) ?
\end{Verbatim}
\end{frame}



\begin{frame}
  \frametitle{La SLD-Résolution}
  $$\frac{\neg(A_{1} \wedge A_{2} \wedge ... \wedge A_{n}) \qquad B_{1} \leftarrow B_{2} \wedge ... \wedge B_{k}}{\neg( B_{2} \wedge ... \wedge B_{k} \wedge A_{2} \wedge ... \wedge A_{n})\theta_{1}}$$

  \vspace{1em}
  Avec $\theta_{1} = \textsc{mgu}(A_{1}, B_{1})$.

  \vspace{1em}

  Si $\neg(A_{1} \wedge ... \wedge A_{n}) \Rightarrow Faux$, alors $(A_{1} \wedge ... \wedge A_{n})\theta_{1}$ est $Vrai$.

  \vspace{1em}

  Unification : substitution des variables de deux termes. Ex : \\
  \texttt{étudiant(E, P) = étudiant(E, pierre)}

  \vspace{1em}

  Il existe un unifieur plus général.

\end{frame}


\begin{frame}[fragile]
\frametitle{Algorithme d'unification}
\begin{columns}
\column{\dimexpr\paperwidth-30pt}
\begin{Verbatim}[fontsize=\footnotesize]
E = {étudiant_de(E,P) = étudiant_de(E, pierre)}.

Répéter tant que E change :
    Sélectionner une équation  s = t dans E;
    Si s = t est de la forme :
        f(s1, ..., sn) = f(t1, ..., tn) avec n >= 0
            Alors remplacer l'équation par s1=t1 ... sn=tn
        f(s1, ..., sm) = g(t1, ..., tn) avec f != g
            Alors ÉCHEC
        X = X
            Alors supprimer l'équation
        t = X où t n'est pas une variable
            Alors remplacer l'équation par X = t
        X = t où X != t et X apparait plus d'une fois dans E
            Si X est un sous-terme de t Alors ÉCHEC
            Sinon on remplace toutes les autres occurences de X par t
\end{Verbatim}
\end{columns}

\vspace{1em}

Termine et est correct.
\end{frame}


\begin{frame}
  \frametitle{Le retour sur trace}
  \begin{columns}

  \column{0.5\textwidth}
  \begin{center}
  \begin{tikzpicture}[sibling distance=3em, level distance=3em,
    every node/.style = {shape=circle, draw=black, fill=black, align=center},
    level 2/.style={sibling distance=6em},
    level 3/.style={sibling distance=3em}]
    \node (a) at (0, 0) {}
    child { node (b) {} }
    child { node (c) {}
      child { node (d) {}
        child { node (e) {} }
        child { node (f) {} }
       }
       child { node (g) {}
         child { node (h) {} }
         child { node (i) {} }  }};

     \draw[->,>= latex, purple] (a) to[bend left=-20] (b);
     \draw[->,>= latex, purple] (b) to[bend left=-20] (a);
     \draw[->,>= latex, purple] (a) to[bend left=-20] (c);
     \draw[->,>= latex, purple] (c) to[bend left=-20] (d);
     \draw[->,>= latex, purple] (d) to[bend left=-20] (e);
     \draw[->,>= latex, purple] (e) to[bend left=-20] (d);
     \draw[->,>= latex, purple] (d) to[bend left=-20] (f);
     \draw[->,>= latex, purple] (f) to[bend left=-20] (d);
     \draw[->,>= latex, purple] (d) to[bend left=-20] (c);
     \draw[->,>= latex, purple] (c) to[bend left=-20] (g);
     \draw[->,>= latex, purple] (g) to[bend left=-20] (h);
     \draw[->,>= latex, purple] (h) to[bend left=-20] (g);
     \draw[->,>= latex, purple] (g) to[bend left=-20] (i);
     \draw[->,>= latex, purple] (i) to[bend left=-20] (g);
     \draw[->,>= latex, purple] (g) to[bend left=-20] (c);
     \draw[->,>= latex, purple] (c) to[bend left=-20] (a);
  \end{tikzpicture}
\end{center}

  \column{0.5\textwidth}
  Il faut choisir dans quel ordre utiliser les clauses !

  \vspace{1em}

  Solution utilisée en Prolog : parcours en profondeur.

  \begin{exampleblock}{Avantage}
    Faible cout en mémoire.
  \end{exampleblock}

  \end{columns}
\end{frame}



\begin{frame}
  \frametitle{Le retour sur trace}
  \begin{columns}

  \column{0.5\textwidth}
  \begin{center}
  \begin{tikzpicture}[sibling distance=3em, level distance=3em,
    every node/.style = {shape=circle, draw=black, fill=black, align=center},
    level 2/.style={sibling distance=6em},
    level 3/.style={sibling distance=3em}]
    \node (a) at (0, 0) {}
    child { node (b) {} }
    child { node (c) {}
      child { node (d) {}
        child { node (e) {} }
        child { node (f) {} child {node (j) {} child {node[fill=white, draw = white] (k) {...}}} }
       }
       child { node (g) {}
         child { node (h) {} }
         child { node (i) {} }  }};

     \draw[->,>= latex, purple] (a) to[bend left=-20] (b);
     \draw[->,>= latex, purple] (b) to[bend left=-20] (a);
     \draw[->,>= latex, purple] (a) to[bend left=-20] (c);
     \draw[->,>= latex, purple] (c) to[bend left=-20] (d);
     \draw[->,>= latex, purple] (d) to[bend left=-20] (e);
     \draw[->,>= latex, purple] (e) to[bend left=-20] (d);
     \draw[->,>= latex, purple] (d) to[bend left=-20] (f);
     \draw[->,>= latex, purple] (f) to[bend left=-20] (j);
     \draw[->,>= latex, purple] (j) to[bend left=-20] (k);
    
  \end{tikzpicture}
\end{center}

  \column{0.5\textwidth}
  Il faut choisir dans quel ordre utiliser les clauses !

  \vspace{1em}
 
  Solution utilisée en Prolog : parcours en profondeur.

  \begin{exampleblock}{Avantage}
    Faible cout en mémoire.
  \end{exampleblock}

  \begin{alertblock}{Inconvénient}
    Risque manquer des solutions (boucles infinies).
  \end{alertblock}
 
  \end{columns}
\end{frame}


\end{document}
